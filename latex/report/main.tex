\documentclass[paper=a4, DIV=10, fontsize=12pt, parskip=full]{scrartcl}

% font
\usepackage[utf8]{inputenc}
\usepackage[T1]{fontenc}
\usepackage{kpfonts}
\usepackage[scaled=0.95]{inconsolata}

% typography
\setkomafont{subject}{\normalfont\bfseries\large}
\setkomafont{title}{\normalfont\bfseries}
\setkomafont{subtitle}{\normalfont\bfseries\large}
\setkomafont{author}{\normalfont\normalsize}
\setkomafont{date}{\normalfont\bfseries\normalsize}

\setkomafont{disposition}{\normalfont\bfseries}
\setkomafont{section}{\normalsize}

% minted
\usepackage{minted}
\setminted{style=stata-light,bgcolor=gray!5,linenos=true}

% commands
\newcommand{\question}[3]{\section*{\underbar{Question #1 [#2 points]}}\vspace{-1em}\textit{#3}}
\newcommand{\subquestion}[2]{\textit{#1) #2}}
\newcommand{\icode}[1]{\mintinline{stata}{#1}}

%headers
\usepackage{scrlayer-scrpage}
\clearpairofpagestyles
\ihead{FEM11087}
\chead{Group 23}
\ohead{Fall 25}
\cfoot{\pagemark}

%title
\subject{FEM 11087 - Applied Microeconometrics}
\title{\Large{Assignment 1: Empirical Analysis}}
\subtitle{Regression Analysis with Cross-Sectional Data, Endogeneity and Instrumental Variable Estimation}
\author{\textbf{Group 23}\\[0.5em]
    Kees-Piet Barnhoorn\\
    Tyler McGee\\
    Andres Pinon\\
    Jolien Schaeffers}
\date{16 September 2025}

\begin{document}

\maketitle

% question one
\question{1}{0.6}{First generate the variable BMI, where BMI equals weight in kg divided by height in meters squared (BMI = weight/(height²)). Construct a categorical variable for BMI that considers the commonly used categories: i) underweight, BMI below 18.5; ii) normal weight, BMI larger or equal to 18.5 and lower than 25; iii) overweight, BMI larger or equal to 25 and lower than 30; iv) obese, BMI of 30 or higher. Compute and report the prevalence of overweight and obesity by ethnic group (black vs non-black). What differences do you observe?}



\subquestion{a}{Compute and report the prevalence of overweight and obesity by ethnic group (black vs non-black). What differences do you observe?}



\end{document}