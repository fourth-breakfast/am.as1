% question two
\question{2}{0.9 points}

\note{Estimate a multivariate regression model explaining income as a function of BMI and whether the individual is black. Thus, estimate: \begin{equation}income= \beta_0 + \beta_1BMI + \beta_2black + u\end{equation}}

Using \icode{reg} with robust SE estimators:

\begin{minted}{stata}
reg income bmi black, robust
\end{minted}

\begin{minted}{stata}
Linear regression                               Number of obs     =        445
                                                F(2, 442)         =       1.07
                                                Prob > F          =     0.3450
                                                R-squared         =     0.0047
                                                Root MSE          =      39216

------------------------------------------------------------------------------
             |               Robust
      income | Coefficient  std. err.      t    P>|t|     [95% conf. interval]
-------------+----------------------------------------------------------------
         bmi |  -307.3532   277.6654    -1.11   0.269    -853.0618    238.3554
       black |  -4839.756   5108.111    -0.95   0.344    -14878.96    5199.447
       _cons |   51459.45   8992.888     5.72   0.000     33785.32    69133.59
------------------------------------------------------------------------------    
\end{minted}

\subquestion{a}{Interpret the estimated coefficients of all the explanatory variables (\textbf{sign, magnitude, and significance}).}

For the explanatory variable \icode{bmi}, the negative coefficient implies a negative relationship. The magnitude of the coefficient, 307.3532, implies that for each unit increase in BMI, an individual's yearly income would decrease by 307\$, ceteris paribus. Finally, it can be determined that the \icode{bmi} variable is not significant, as its p-value (0.269) exceeds our significance level (0.05).

In the case of the variable \icode{black}, the negative coefficient implies again a negative relationship with the dependent variable. The magnitude of the coefficient, 4839.756, implies a decrease of roughly 4840\$ \space in annual income if the individual belongs to the black ethnic group, ceteris paribus. As in the case of \icode{bmi}, \icode{black}'s p-value (0.344) exceeds the significance level, meaning it is not significant.

\subquestion{b}{What is the estimated 95\% Confidence Interval of $\beta_1$? What can you conclude based on the information in this confidence interval about the \textbf{effect of BMI on income}?}

As seen in the regression above, the estimated 95\% CI of $\beta_1$ is [-788.01, 286.60]. Since it contains 0, we cannot reject the null hypothesis that BMI has no effect on income at the 5\% significance level. This aligns with the conclusions we derived from the p-values in the previous question.

\subquestion{c}{What is the \textbf{relative magnitude} of the effect of being black on income? Interpret the relative magnitude.}

The relative magnitude of being black on income can be explained as the \% change in income experienced if an individual belongs to the black ethnic group, ceteris paribus. We can calculate it as:

\vspace{-1em}

\begin{equation}
       \text{Relative Magnitude} = \text{Coefficient} / \text{Mean Income}
\end{equation}

\vspace{-1em}

\begin{equation}
       -4839.75 / 41,514.35 = -0.1165
\end{equation}

We can therefore determine that the coefficient for black ethnicity represents roughly 11.65\% of mean annual income.