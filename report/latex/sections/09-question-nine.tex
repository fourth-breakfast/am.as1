% question nine
\question{9}{0.7 points}

\vspace{-1em}

\subquestion{a}{Draw a \textbf{DAG} that illustrates the assumptions required for \icode{drinks} to be a good instrument for BMI.}

\subquestion{b}{Explain these assumptions in your own words.}

The four assumptions that are required for drinks to be a good instrument are as follows: 

\begin{itemize}
    \item Omitted Variable Bias: Unobserved factors exist which could influence both the dependent and explanatory variable in the model. 
    \item Relevance: A relevant instrument is one in which it is correlated with the endogenous variable such as o Cov(x,z) != 0, where x is the endogenous variable and z is the instrumental variable. 
    \item Validity: A valid instrumental variable is also one that is uncorrelated with the unobserved factors of the model so that o Cov(u,z) = 0, where u is the unobserved variables and z is the instrumental variable. 
    \item Additionally, the instrumental variable must only affect the dependent variable through the endogenous variable, it is not allowed to directly affect the dependent variable.
\end{itemize}

\newpage

\subquestion{c}{In your opinion, do these assumptions hold? Discuss without any further analysis. You may use evidence from previous questions.}

With regards to the analyses, we've previously conducted; we can safely say that we believe that the omitted variable bias does hold. This is supported by the RESET tests conducted in question 5 which proved that models already rejected the null hypothesis that there were no omitted variables. We believe that the additional instrumental variable in drinks is not the only omitted variable that has an effect, and there could be additional variables such as health status that are not included in the model which affect our dependent and independent variables.

For the relevance assumption: we agree that the instrumental variable drinks could very well be influencing BMI scores, considering the regression outputs we conducted and from the health hazards that consuming large amounts of sugar can have. We do not agree however, with the claim that it is a strong instrument, as drinks are only one item that causes BMI levels to change; and there are so many other food items and activities that play a huge factor in determining the level of BMI of an individual. 

For the validity assumption: We believe that this assumption would not hold up for the instrumental variable drinks as there are multiple different ways that the consumption of sweet drinks influences the level of income in an individual. An example of this would stem from level of productivity. Sweet drinks (that contain sugar) are commonly used to feel more energized which allows individuals to get more work done;this work productivity can definitely have an effect on the level of annual income as more productivity individuals are believed to have higher incomes than those who are not as productive.