% question eleven

\question{11}{0.4 points}

\note{Researchers want to estimate the effect of \textbf{peers' risky health behaviors} (alcohol consumption) on the \textbf{academic performance of adolescents} by exploiting the quasi-exogenous assignment of high-school students across classes. They want to use the \textbf{peers' fathers' drinking behavior} as an instrumental variable for peers' risky health behaviors. They estimate a linear regression model with peers' risky health behaviors as the dependent variable (measured as the average number of times peers consumed alcohol in the past month) and the peers' fathers' drinking as an explanatory variable. The coefficient of this explanatory variable is \textbf{positive and statistically significant}.}

\note{Based on this information, what is your \textbf{assessment} of this identification strategy to estimate the
effect of peers’ risky health behaviors on the academic performance of adolescents?}

The first-stage regression results indicate that peers' fathers' drinking behavior is positively and significantly correlated with peers' risky alcohol consumption, supporting the relevance criterion for a valid instrument. This suggests that variation in peers' fathers' drinking can explain variation in peers' risky behaviors, which is essential for identifying the causal effect of interest. 

However, the validity of the instrumental variable critically hinges on the exclusion restriction, which requires that peers' fathers' drinking affects adolescents' academic performance only through peers' risky health behaviors and not via any other direct or indirect pathways. This assumption is difficult to verify and potentially problematic. Peers' fathers' drinking may be correlated with broader family or community characteristics, such as socioeconomic status, parental involvement, or cultural norms, that could independently influence academic outcomes. Without convincing evidence that these confounding channels are ruled out or adequately controlled for, the exclusion restriction remains questionable. 

Finally, the success of the identification strategy also depends on the quasi-exogeneity of student assignment across classes. If the assignment process is truly random or as good as random, it strengthens the plausibility that the instrument isolates exogenous variation in peer behavior. Conversely, if sorting occurs based on unobserved factors related to both peers' fathers' drinking and academic performance, the instrument may be endogenous. Overall, while the strategy is well-thought out and has potential.